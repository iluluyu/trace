\documentclass[tikz, border=2pt]{standalone}
\usepackage{tikz}
\usepackage{amssymb} 
\usetikzlibrary{arrows.meta, positioning, calc}

% --- 配色方案 ---
\definecolor{WLTeal}{HTML}{00897B}   
\definecolor{WLOrange}{HTML}{FF7043} 
\definecolor{WLViolet}{HTML}{5E35B1} 
\definecolor{WLGray}{HTML}{546E7A}   

\begin{document}

\tikzset{
    global style/.style={
        node distance=0.15cm, % 球与球之间的紧凑间距保持不变
        font=\sffamily
    },
    ball/.style={
        circle, 
        minimum size=0.65cm, 
        inner sep=0pt, 
        draw=none
    },
    blueball/.style={ball, fill=WLTeal},
    redball/.style={ball, fill=WLOrange},
    purpleball/.style={ball, fill=WLViolet}, 
    boxstyle/.style={
        draw=WLGray!60, 
        dashed, 
        line width=1pt,
        rounded corners=4pt,
        minimum width=0.9cm,  % <--- 修改点1:缩小方框宽度 (原1.4cm)
        minimum height=0.9cm, % <--- 修改点2:稍微减小高度 (原1.0cm)
        fill=gray!5,
        align=center
    },
    labeltext/.style={font=\scriptsize\sffamily, text=WLGray, align=center},
    titletext/.style={font=\small\bfseries\sffamily, text=WLGray!90, align=center}
}

\begin{tikzpicture}[global style]

    % ============================================
    % 左侧: Case 1 Distinguishable
    % ============================================
    \begin{scope}[local bounding box=case1]
        % 1. 标题
        \node[titletext] at (0, 1.0) {Case 1: Distinguishable ($d=n$)};

        % 2. 中间:算符 (Box)
        \node[boxstyle] (sigma1) at (0,0) {};
        \node[text=WLOrange, font=\large\bfseries] at (sigma1.center) {$\sigma$};

        % 3. 左侧输入 (青, 紫, 橙)
        % <--- 修改点3:增加方框与球的距离 (0.2cm -> 0.5cm),让箭头更长更明显
        \node[redball]    (L1_3) [left=0.5cm of sigma1] {}; 
        \node[purpleball] (L1_2) [left=of L1_3] {};
        \node[blueball]   (L1_1) [left=of L1_2] {};
        
        % 箭头
        \draw[->, >={Stealth[scale=0.8]}, line width=1pt, WLGray!70] (L1_3.east) -- (sigma1.west);

        % 4. 右侧输出 (青, 橙, 紫)
        % <--- 修改点4:同理增加右侧距离
        \node[blueball]   (R1_1) [right=0.5cm of sigma1] {};
        \node[redball]    (R1_2) [right=of R1_1] {};    
        \node[purpleball] (R1_3) [right=of R1_2] {};    
        
        % 箭头
        \draw[->, >={Stealth[scale=0.8]}, line width=1pt, WLGray!70] (sigma1.east) -- (R1_1.west);

        % 5. 底部结论
        \node[labeltext, below=0.2cm of sigma1] {Unique Parity \checkmark};
    \end{scope}


    % ============================================
    % 右侧: Case 2 Indistinguishable
    % ============================================
    \begin{scope}[xshift=7.5cm, local bounding box=case2] 
        % 1. 标题
        \node[titletext] at (0, 1.0) {Case 2: Indistinguishable ($d < n$)};

        % 2. 中间:算符
        \node[boxstyle] (sigma2) at (0,0) {};
        \node[text=WLOrange, font=\large\bfseries] at (sigma2.center) {$\sigma'$};

        % 3. 左侧输入 (青, 青, 橙)
        % <--- 修改点3:距离增加至 0.5cm
        \node[redball]  (L2_3) [left=0.5cm of sigma2] {};
        \node[blueball] (L2_2) [left=of L2_3] {};
        \node[blueball] (L2_1) [left=of L2_2] {};
        
        \draw[->, >={Stealth[scale=0.8]}, line width=1pt, WLGray!70] (L2_3.east) -- (sigma2.west);

        % 4. 右侧输出 (青, 青, 橙)
        % <--- 修改点4:距离增加至 0.5cm
        \node[blueball] (R2_1) [right=0.5cm of sigma2] {};
        \node[blueball] (R2_2) [right=of R2_1] {};
        \node[redball]  (R2_3) [right=of R2_2] {};
        
        \draw[->, >={Stealth[scale=0.8]}, line width=1pt, WLGray!70] (sigma2.east) -- (R2_1.west);

        % 5. 底部结论
        \node[labeltext, below=0.2cm of sigma2] {Ambiguous Parity ?\\(Even or Odd?)};
    \end{scope}

    % 中间分割线
    \draw[WLGray!20, line width=1pt] (3.75, 1.2) -- (3.75, -1.0);

\end{tikzpicture}
\end{document}