\documentclass[aspectratio=169]{beamer}
\usetheme{WritingLab}

% Packages
\usepackage{physics}
\usepackage{braket}
\usepackage{tikz}
\usepackage{amsmath}
\usepackage{amssymb}
\usepackage{bm}
\usepackage{dsfont}
\usetikzlibrary{shapes.geometric, arrows.meta, positioning, calc, shadows, fadings}

% Metadata
% 1. 标题:保持你的手动换行,视觉平衡
\title{Quantum Advantage in  \\ Identifying the Parity of Permutations with Certainty}

\subtitle{
    \texorpdfstring{%
        % --- 页面上显示的内容 (带颜色、换行、加粗) ---
        \textbf{\textcolor{TitleBlue}{Phys. Rev. Lett. (Published: Dec 23, 2025)}} \\
        \vspace{0.3em}
        \textcolor{WLGray}{Preprint: arXiv:2508.04310 [quant-ph] (Aug 6, 2025)}
    }{%
        % --- PDF 属性里记录的内容 (纯文本,防止报错) ---
        Phys. Rev. Lett. (2025) - arXiv:2508.04310
    }
}

% 3. 作者:这里列出【论文原作者】
% 技巧:人名较多时,使用 \and 或者手动换行 \\ 来控制排版
\author{
    A. Diebra\inst{1}, S. Llorens\inst{1}, D. González-Lociga\inst{1}, A. Rico\inst{1}, \\
    J. Calsamiglia\inst{1}, M. Hillery\inst{2}, and E. Bagan\inst{1}
}

% 4. 机构:这里列出【原作者单位】
% 如果你是汇报人,通常不需要把复旦大学放在这里抢占位置,
% 除非你想强调是在复旦大学的组会上汇报。
\institute{
    \inst{1} Universitat Autònoma de Barcelona (UAB) \\
    \inst{2} Hunter College of CUNY
}

% 5. 日期:【最关键的修改】
% 不要用 \today,而是写文献的具体日期和版本
% 建议格式:[版本号] + [日期]
\date{\today}

% 6. (可选) 如果你想让 Title Page 更酷,加个 arXiv 的 logo 或者二维码
% 需要 \usepackage{qrcode}
% \titlegraphic{\qrcode[height=1.5cm]{https://arxiv.org/abs/2508.04310}}

\begin{document}

% --- Slide 1: Title ---
\begin{frame}[plain, noframenumbering]
    \titlepage
\end{frame}

% --- Slide 2: Context ---
\begin{frame}{Introduction: The Quest for Quantum Advantage}
    \begin{stylebox}[6]{Context}
        Quantum Information Science seeks tasks where quantum systems fundamentally outperform classical ones.
    \end{stylebox}

    \begin{stylebox}[5]{This Work's Focus}
        A physically natural problem involving \textbf{permutations of particles}.
        \begin{itemize}
            \item No Oracles.
            \item No Computational Promises.
            \item \textbf{Sharp, Non-asymptotic separation.}
        \end{itemize}
    \end{stylebox}
\end{frame}

% --- Slide 3: The Problem Statement ---
\begin{frame}{The Problem: Parity of a Hidden Permutation}
    
    % Placeholder for Problem Setup Diagram
    \begin{center}
        \includegraphics[width=0.8\textwidth]{pic/parity_diagram.pdf}
    \end{center}

    \begin{stylebox}[1]{Problem Definition}
        Given a set of $n$ particles subject to a hidden permutation $\sigma \in S_n$, determine the parity:
        \begin{itemize}
            \item Is $\sigma$ \textbf{Even}? (e.g., identity, 3-cycles)
            \item Is $\sigma$ \textbf{Odd}? \phantom{e} (e.g., single swap)
        \end{itemize}
    \end{stylebox}

    \begin{eqbox}[3]
        \textbf{Goal:} Identify parity with \textbf{Certainty} ($P_s = 1$).
    \end{eqbox}
\end{frame}

% --- Slide 4: Classical Limitation ---
\begin{frame}{The Classical Limit}
        % 3. 插图区域 (你要求的代码)
    % 这里展示 d < n 时的不可区分性 (Indistinguishability)
    \begin{center}
        \includegraphics[width=0.9\textwidth]{pic/parity_comparison.pdf}
    \end{center}
    \vspace{-1.5em}
    \begin{columns}[T,onlytextwidth]
        \tcbset{equal height group=cl}%
        \begin{column}{0.48\textwidth}
            \begin{ceqbox}[1]
                    Full Distinguishability ($n$ labels)
                    \begin{itemize}
                        \item perfectly reconstruct the permutation
                        \item determine its parity
                        \item $P_s = 1$
                    \end{itemize}              
            \end{ceqbox}
        \end{column}
        \begin{column}{0.48\textwidth} 
            \begin{ceqbox}[5]
                    Insufficient Labels ($< n$ labels)
                                    \begin{itemize}
                    \item perfect identification becomes impossible
                    \item indistinguishable opposite-parity counterparts
                    \item limited to random guessing
                    \item $P_s = 1/2$
                \end{itemize}                       
            \end{ceqbox}
        \end{column}
    \end{columns}
\end{frame}

% --- Slide 5: The Quantum Solution ---
\begin{frame}{Theorem: The Quadratic Quantum Advantage}
    
    \begin{theorem}[Main Result]
        Perfect parity identification ($P_s = 1$) is achievable if
        \vspace{0.2em}
        \begin{equation*}
            d \ge d_{min} := \lceil \sqrt{n} \rceil
        \end{equation*}
        Below this threshold, assuming all permutations are equally likely, parity remains indistinguishable and random guessing is the best one can do ($P_s = 1/2$).
    \end{theorem}

    \begin{tceqbox}[1]{}
        Below $d = \lceil \sqrt{n} \rceil$, even Quantum Mechanics cannot help ($P_s = 1/2$).
    \end{tceqbox}
\end{frame}
% --- Slide: Explicit Example (4 Qubits) ---
\begin{frame}{Example: 4 Qubits ($n=4, d=2$)}
    \begin{stylebox}[1]{Physical Setup}
        Consider a system of \textbf{four spin-1/2 particles}. 
        Alice acts by reshuffling their positions \textit{without} disturbing the spin degrees of freedom.
    \end{stylebox}

    \vspace{-1.5em}

    % 2. 核心分解公式 (使用定理色 TitleBlue)
    % 对应原文 Eq. (3)
    \begin{stylebox}[3]{{Schur-Weyl Decomposition}}
        \begin{align*}
        (\mathbb{C}^2)^{\otimes 4} \cong 
        (\mathcal{K}_2 \otimes \mathcal{H}_2) \oplus 
        (\mathcal{K}_1 \otimes \mathcal{H}_1) \oplus 
        (\mathcal{K}_0 \otimes \mathcal{H}_0)
        \end{align*}
    \end{stylebox}
    \vspace{-1em}
    \begin{columns} 
        \begin{column}{0.45\textwidth}
            \begin{itemize}
                \item $\mathcal{H}_j$: $SU(2)$-invariant subspaces (labeled by total spin $j$).
                \item $SU(2)$ acts \textit{only} on $\mathcal{H}_j$ and trivially on $\mathcal{K}_j$.
            \end{itemize} 
        \end{column}
        \begin{column}{0.45\textwidth} 
            \begin{stylebox}[6]{Multiplicity Spaces ($\mathcal{K}_j$)}
                \begin{itemize}
                    \item $j=2$: $\mathcal{K}_2 \cong \mathbb{C}$ \hfill (1 copy)
                    \item $j=1$: $\mathcal{K}_1 \cong \mathbb{C}^3$ \hfill (3 copies)
                    \item $j=0$: $\mathcal{K}_0 \cong \mathbb{C}^2$ \hfill (2 copies)
                \end{itemize}
            \end{stylebox}
        \end{column}
    \end{columns}
\end{frame}
\begin{frame}{Symmetry and Representation: The $S_4$ Perspective}
        \begin{ceqbox}{\textbf{Permutational Symmetry ($S_4$)}}
        \begin{itemize}
            \item Reshuffling 4 qubits $\longleftrightarrow$ ``Rotation'' within subspaces $K_j$.
            \item These spaces carry \textbf{irreducible representations (irreps)} of $S_4$.
            \item \textit{Note:} No proper subspace is invariant under $\sigma|\psi\rangle$.
        \end{itemize}
        \end{ceqbox}
        \begin{ceqbox}[2]{\textbf{Schur-Weyl Duality Structure}}
            \begin{itemize}
                \item Agent's operations are restricted to permutations.
                \item \textbf{Consequence:} Action is \textbf{trivial} on spin sectors $H_j$.
                \item \textbf{Interpretation:} 
                \begin{itemize}
                    \item $K_j$: Irrep spaces (Dynamic part).
                    \item $H_j$: Multiplicity spaces (Static part).
                \end{itemize}
            \end{itemize}
        \end{ceqbox}

\end{frame}
\begin{frame}{Restriction to Even Permutations ($A_4 \subset S_4$)}
    % 第一部分:群论结构的改变
    \begin{ceqbox}[2]{\textbf{Group Restriction:} Restrict agent actions to \textit{even permutations} ($A_4$).}
        \begin{itemize}
            \item $\mathcal{K}_1$ and $\mathcal{K}_2$: Remain irreducible.
            \item $\mathcal{K}_0$: \textbf{Decomposes} into two orthogonal 1D invariant subspaces:
            \[
                \mathcal{K}_0 \xrightarrow{A_4} \mathcal{K}_{0a} \oplus \mathcal{K}_{0b}
            \]
        \end{itemize}
    \end{ceqbox}
    \vspace{-1.4em}
    \begin{columns}[T,onlytextwidth]
        \tcbset{equal height group=parity}%
        \begin{column}{0.45\textwidth}
                \begin{tceqbox}[1]{Action of Odd Permutations ($\sigma_{\text{odd}} \in S_4 \setminus A_4$)}
                    Any odd permutation swaps the subspaces:
                    \vspace{-0.5em}
                    \[
                        \sigma_{\text{odd}}: \mathcal{K}_{0a} \longleftrightarrow \mathcal{K}_{0b}
                    \]
                    \small{\textit{Note: This explains why $\mathcal{K}_{0a/b}$ are not invariant under the full group $S_4$.}}
                \end{tceqbox}
        \end{column}
        \hfill
        \begin{column}{0.53\textwidth} 
            \begin{tceqbox}[3]{Parity Identification Protocol}
                Prepare initial state $|\psi_e\rangle \in \mathcal{K}_{0a} \otimes \mathcal{H}_0$. The odd/even sectors become orthogonal:
                \begin{equation}
                    \sigma_{\text{odd}}|\psi_e\rangle \perp \sigma_{\text{even}}|\psi_e\rangle
                \end{equation}
                \textbf{Result:} Parity can be identified with certainty via projective measurement.
            \end{tceqbox}   
        \end{column}
    \end{columns}
\end{frame}
% --- Slide: Explicit Construction of the State ---
\begin{frame}{Explicit Construction: The M4 State}
    
    % 1. 构造逻辑:使用青色主题盒 (WLTeal)
    \begin{tceqbox}[1]{Construction via Projection}
        We construct $|\psi_e\rangle$ by projecting the basis state $|0011\rangle$ (with magnetic quantum number $m=0$) onto the invariant subspace $\mathcal{K}_{0a} \otimes \mathcal{H}_0$:
        \vspace{-1em}
        \begin{equation*}
            |\psi_e\rangle = \mathcal{P}_{0a}|0011\rangle, \quad \text{where } \mathcal{P}_{0a} = \frac{1}{|A_4|} \sum_{\sigma \in A_4} \overline{\chi}_{0a}(\sigma)\sigma
        \end{equation*}
    \end{tceqbox}

    \vspace{-1em}

    % 2. 核心公式:使用深色填充盒 (TitleBlue) 强调结果
    \begin{tceqbox}[3]{Computational Basis Expansion:}
        \vspace{-2em}
        \begin{align*}
            |\psi_e\rangle = (|0011\rangle + |1100\rangle) + \zeta_3 \phantom{^1} (|0101\rangle + |1010\rangle) + \zeta_3^2 \phantom{^1} (|0110\rangle + |1001\rangle)
        \end{align*}
    \end{tceqbox}

    \vspace{-1.5em}

    % 3. 补充说明:双栏布局
    \begin{columns}[T,onlytextwidth]
        \tcbset{equal height group=M4}%
        \begin{column}{0.65\textwidth}
            % 左侧:物理意义 (灰色/引用色)
            \begin{tceqbox}[6]{}
                \begin{itemize}
                    \item Also known as the \textbf{M4 (or D4) state}.
                    \item A Fourier-type combination of \textbf{Dicke states}.
                \end{itemize}
            \end{tceqbox}
        \end{column}
        \begin{column}{0.3\textwidth}
            % 右侧:数学细节 (绿色/算法色)
            \begin{tceqbox}[2]{}
                $\zeta_3 = e^{i 2\pi/3}$ (Cubic root of unity).
            \end{tceqbox}
        \end{column}
    \end{columns}
\end{frame}
% --- Slide: Young Diagrams Labeling ---
\begin{frame}{Labeling Invariant Subspaces: Young Diagrams}
    \vspace{0.2em}
    \begin{center}
        \includegraphics[width=0.8\linewidth]{pic/young_diagrams.pdf}
    \end{center}
    \vspace{-1.2em}
    \begin{columns}[T, totalwidth=\textwidth]
        \tcbset{equal height group=YD}%
        \begin{column}{0.48\textwidth}
            % 左侧:物理意义 (灰色/引用色)
            \begin{stylebox}[1]{Partitions \& Young Diagrams (YDs)}
                Subspaces are labeled by partitions $\lambda \vdash n$, visualized as YDs with $n$ boxes.
                \begin{gather*}
                    \lambda = [\lambda_1, \lambda_2, \dots] \\
                    (\lambda_1 \ge \lambda_2 \ge \dots \ge 0, \;\; \sum \lambda_p = n)
                \end{gather*}
            \end{stylebox}
        \end{column}
        \begin{column}{0.48\textwidth}
            \begin{outlinebox}[WLBlue]
                \begin{itemize}
                    \item \textbf{$S_n$ Irreps ($\mathcal{K}_\lambda$):} 
                    Correspond to \textit{any} distinct $\lambda \vdash n$.
                    \item \textbf{$SU(d)$ Irreps ($\mathcal{H}_\lambda$):} 
                    Restricted by dimension $d$:
                    \[
                        \ell(\lambda) \le d
                    \]
                    (The YD cannot have more than $d$ rows)
                \end{itemize}
            \end{outlinebox}
        \end{column}
    \end{columns}
\end{frame}

% --- Slide: Young Diagrams and Subspace Labeling ---
\begin{frame}{Labeling Subspaces: Young Diagrams (YDs)}
    \begin{center}
        \includegraphics[width=0.9\linewidth]{pic/young_diagrams.pdf}
    \end{center}
    \vspace{-1.5em}
    % 用灰色盒子列出图对应的分划,作为图注
    \begin{outlinebox}[WLGray]
        \centering \textbf{Partitions of $n=4$}\\
        \footnotesize
        $[4], \quad [3,1], \quad [2,2], \quad [2,1,1], \quad [1^4]$
    \end{outlinebox}
    \vspace{-0.5em}
    \begin{columns}[T,totalwidth=\textwidth]
        % \tcbset{equal height group=k31}%
        \begin{column}{0.60\textwidth}
            % n=4 的特殊情况 (使用强调色 Orange - 对应图片2的关键点)
            \begin{tceqbox}[5]{Crucial Subspaces for $n=4$}
                For $d \ge 4$, all 5 irreps appear. 
                \begin{itemize}
                    \item $\mathcal{K}_{[4]}$: \textbf{Fully Symmetric} (Trivial, $+1$).
                    \item $\mathcal{K}_{[1^4]}$: \textbf{Fully Antisymmetric} (Sign, $\pm 1$).
                \end{itemize}
                \small{\textit{Mixing these enables perfect parity identification.}}
            \end{tceqbox}
        \end{column}
        \hfill
        \begin{column}{0.38\textwidth}
            \begin{tceqbox}[4]{}
                \vspace{-1.5em}
                \begin{align*}
                    |\psi_e\rangle = |\psi_{\text{sym}}\rangle + |\psi_{\text{ant}}\rangle
                \end{align*}
            \end{tceqbox}
            \begin{tceqbox}[2]{}
                \vspace{-1.5em}
                \begin{align*}
                    |\psi_{\text{sym}}\rangle &\in \mathcal{K}_{[4]} \otimes \mathcal{H}_{[4]} \\
                    |\psi_{\text{ant}}\rangle &\in \mathcal{K}_{[1^4]} \otimes \mathcal{H}_{[1^4]}
                \end{align*}
                \vspace{-1.8em}
            \end{tceqbox}
        \end{column}
    \end{columns}



\end{frame}
\begin{frame}{Case 2: Restricted Dimensions ($d = 3$)}
    \begin{columns}[T,totalwidth=\textwidth]
        % \tcbset{equal height group=k31}%
        \begin{column}{0.70\textwidth}
            \begin{tceqbox}[5]{The Constraint Problem($d=3$)}
                Fully antisymmetric partition $[1^4]$ is \textbf{forbidden}.
                \vspace{-1em}
                \[
                    \ell([1^4]) = 4 > d=3 \implies |\psi_{\text{ant}}\rangle \text{ does not exist.}
                \]
            \end{tceqbox}
        \end{column}
        \hfill
        \begin{column}{0.28\textwidth}
            \begin{center}
                \includegraphics[width=0.9\linewidth]{pic/yd43.pdf}
            \end{center}
        \end{column}
    \end{columns}
    
    % 1. 问题提出 (橙色警告)

    % 2. 解决方案:共轭对 (蓝色定理)
    \begin{stylebox}[3]{Solution: Conjugate Pairs}
        We use conjugate partitions $\lambda$ and $\lambda^T$ (reflecting YD along diagonal).
        They satisfy a crucial representation property:
        \begin{equation*}
             D_{\lambda^T}(\sigma) = \text{sign}(\sigma) D_{\lambda}(\sigma)
        \end{equation*}
    \end{stylebox}
    \vspace{-1.5em}
    % 3. 具体构造
    \begin{columns}[T,totalwidth=\textwidth]
        \tcbset{equal height group=NC}%
        \begin{column}{0.55\textwidth}
            \begin{stylebox}[1]{New Construction:}
                Combine partitions $\lambda=[3,1]$ and $\lambda^T=[2,1^2]$:
                \vspace{-0.7em}
                \[
                    |\psi_e\rangle = |\psi_{[3,1]}\rangle + |\psi_{[2,1^2]}\rangle
                \]
            \end{stylebox}
        \end{column}
        \begin{column}{0.42\textwidth}
            \begin{outlinebox}[WLGray]
                \footnotesize
                Even though we lack the trivial/sign singlets, these higher-dimensional irreps mimic the sign-flipping behavior required for orthogonality.
            \end{outlinebox}
        \end{column}
    \end{columns}
\end{frame}
\begin{frame}{Case 3: Self-Conjugate Irreps \& The General Rule}
    % 1. 问题提出 (橙色警告)
    \begin{tceqbox}[1]{Definition: Self-Conjugate Partitions}
        Its Young Diagram is symmetric along the main diagonal ($\lambda = \lambda^T$).
        \begin{itemize}
            \item \textbf{Example ($n=4$):} $\lambda = [2,2]$ (The square diagram).
        \end{itemize}
    \end{tceqbox}
    \vspace{-1.0em}
    % 2. 解决方案:共轭对 (蓝色定理)
    \begin{columns}[T]
        \begin{column}{0.48\textwidth}
            \begin{tceqbox}[3]{Mechanism: Splitting under $A_n$}
                Unlike generic irreps, self-conjugate ones split when restricted to even permutations ($A_n$):
                \begin{align*}
                    \mathcal{K}_\lambda \xrightarrow{\quad A_n \quad} \mathcal{K}_{\lambda a} \oplus \mathcal{K}_{\lambda b}
                \end{align*}
            \end{tceqbox}   
        \end{column}
        \begin{column}{0.45\textwidth}
            \begin{center}
                \includegraphics[width=0.22\linewidth]{pic/yd42.pdf}
            \end{center}
            \vspace{-1.5em}
            \begin{tceqbox}[3]{}
                Action of Odd Permutations:
                \[
                    \mathcal{K}_{\lambda a} \xrightarrow{\quad \sigma_{\text{odd}} \quad} \mathcal{K}_{\lambda b}
                \]
                % \small{\textit{Result: The sectors became orthogonal, enabling perfect parity distinction.}}
            \end{tceqbox}

        \end{column}
    \end{columns}
\end{frame}
\begin{frame}{Example: Five Qutrits ($d=3$) \& Non-Trivial Coefficients}
    \begin{columns}[T,totalwidth=\textwidth]
        \tcbset{equal height group=k31}%
        \begin{column}{0.53\textwidth}
            % 1. 背景设定 (对应第一张图的上半部分)
            % 使用样式 [1] (Definition/Orange)
            \begin{tceqbox}[1]{Setup: Self-Conjugate Irrep $\mathcal{K}_{[3,1^2]}$}
                Consider $(\mathbb{C}^3)^{\otimes 5}$. The partition $\lambda = [3,1^2]$ is self-conjugate with dimension $d_{[3,1^2]} = 6$.
                \begin{itemize}
                    \item Parity can be identified with certainty using any state $|\psi_e\rangle$ in the invariant subspace $\mathcal{K}_{[3,1^2]a} \otimes \mathcal{H}_{[3,1^2]}$.
                \end{itemize}
            \end{tceqbox}
        \end{column}
        \hfill
        \begin{column}{0.45\textwidth}
            % 2. 投影算符公式 (对应第一张图的公式 9)
            % 使用样式 [3] (Mechanism/Blue)
            \begin{tceqbox}[3]{Mechanism: Projector Construction}
                The projector $\mathcal{P}_{[3,1^2]a}$ onto the subspace (analogous to $\mathcal{P}_{0a}$) is given by:
                \[
                    \mathcal{P}_{\lambda a} = \frac{d_\lambda/2}{|A_n|} \sum_{\sigma \in A_n} \bar{\chi}_{\lambda a}(\sigma) \sigma
                \]
            \end{tceqbox}
        \end{column}
    \end{columns}
    % 3. 结果展示 (对应第二张图的公式 10)
    % 展示非平庸系数,说明 A5 不可解
    \begin{fceqbox}[3]{}
        \vspace{-0.5em}
        \scriptsize % 缩小字号以适应长公式
        \begin{align*}
            |\psi_e\rangle &= 3\big(|00012\rangle - |00021\rangle\big) 
            - |00102\rangle + |00120\rangle + |00201\rangle - |00210\rangle \\
            &\quad - |01002\rangle + |01020\rangle - |10002\rangle + |10020\rangle + \dots \\
            &\quad + \sqrt{5}\big(|01200\rangle - |02100\rangle - |10200\rangle + |12000\rangle + \dots \big)
        \end{align*}
        \vspace{-1em}
    \end{fceqbox}

\end{frame}
% ------------------------------------------------------------------------
% Slide 1: Setup and Schur-Weyl Decomposition
% ------------------------------------------------------------------------
\begin{frame}{Proof (I): Hypothesis Testing \& Schur-Weyl Decomposition}
    \begin{columns}[T,totalwidth=\textwidth]
        
        \begin{column}{0.48\textwidth}
            % 1. Problem Formulation
            \begin{stylebox}[4]{Binary Discrimination Problem}
                Distinguish between even ($\rho_0$) and odd ($\rho_1$) mixtures of permutations.
                \vspace{-1em}
                \begin{align*}
                    \rho_0 &= \frac{1}{|A_n|} \sum_{\sigma \in A_n} \sigma \ket{\psi_e}\bra{\psi_e} \sigma^{-1} \\ \rho_1 &= (12)\rho_0(12)
                \end{align*}
                \vspace{-1.2em}
                \begin{itemize}
                \item Perfect discrimination requires orthogonal supports: $\text{Tr}(\rho_0 \rho_1) = 0$.
                \item If $\rho_0 = \rho_1$, then $P_s = 1/2$ (random guessing).
            \end{itemize}
            \end{stylebox}
        \end{column}
        \hfill
        \begin{column}{0.48\textwidth}
            \begin{stylebox}[5]{Schur-Weyl Decomposition}
                The Hilbert space decomposes into sectors labeled by partitions $\lambda \vdash n$:
                \vspace{-0.5em}
                \begin{equation*}
                    (\mathbb{C}^d)^{\otimes n} \cong \bigoplus_{\lambda \vdash n, \ell(\lambda) \le d} \mathcal{K}_\lambda \otimes \mathcal{H}_\lambda
                \end{equation*}
            \end{stylebox}
            \begin{stylebox}[5]{Partition $\lambda$ Into 3 Sectors}
                    \begin{enumerate}
                        \item $\lambda = \lambda^T$ (Self-conjugate).
                        \item $\lambda \neq \lambda^T$ and $\ell(\lambda^T) \le d$ (Conjugate pairs).
                        \item $\ell(\lambda^T) > d$ (Restricted sector).
                    \end{enumerate}
            \end{stylebox}
        \end{column}
    \end{columns}
\end{frame}


% ------------------------------------------------------------------------
% Slide 2: Block-Diagonal Structure under A_n
% ------------------------------------------------------------------------
\begin{frame}{Proof (II): Action of $A_n$ and Block Structure}
    \begin{lemma}[Block-diagonal Form]
        Under the restriction to the Alternating Group $A_n$, the irreps split differently in each sector. By Schur's lemma, $\rho_0$ takes the block-diagonal form:
        \vspace{-1em}
        $$ \rho_0 = \rho_0^{(i)} + \rho_0^{(ii)} + \rho_0^{(iii)} $$
    \end{lemma}

    \begin{oceqbox}[2]{\smartball[WLTeal]{1} [$\lambda = \lambda^T$]:} Irreps split $\mathcal{K}_\lambda \downarrow_{A_n} = \mathcal{K}_{\lambda a} \oplus \mathcal{K}_{\lambda b}$.
    \begin{equation*}
        \rho_0^{(\text{i})} = \sum_{\lambda \in R_{\text{i}}} \left( \frac{\mathds{1}_{\lambda a}}{d_\lambda/2} \otimes \Phi^{\lambda a} + \frac{\mathds{1}_{\lambda b}}{d_\lambda/2} \otimes \Phi^{\lambda b} \right)
    \end{equation*}
    \end{oceqbox}
    \begin{outlinebox}[WLBlue]
        \begin{center}
        If either $\Phi^{\lambda a} = 0$ or $\Phi^{\lambda b} = 0$, then $\rho_0^{(1)}$ and $\rho_1^{(1)}$ have orthogonal supports;
        \end{center}
    \end{outlinebox}
\end{frame}
\begin{frame}{Proof (II): Action of $A_n$ and Block Structure}

    % 保持你原有的 Box 结构
    \begin{oceqbox}[2]{\smartball[WLTeal]{2} [$\lambda \neq \lambda^T$]:} Irreps appear in conjugate pairs, $(\mathcal{K}_\lambda \oplus \mathcal{K}_{\lambda^T}) \downarrow_{A_n} = \mathcal{K}_\lambda \otimes \mathbb{C}^2$.
        \vspace{-1.2em}
        \begin{equation*}
            \rho_0^{(\text{ii})} = \sum_{\lambda \in R_{\text{ii}}} \frac{\mathds{1}_\lambda}{d_\lambda} \otimes \sum_{k=1}^{d_\lambda} \ket{\phi_k^\lambda}\bra{\phi_k^\lambda}, \quad \ket{\phi_k^\lambda} \in \mathbb{C}^2 \otimes \mathcal{H}_\lambda
        \end{equation*}
    \end{oceqbox}
    \vspace{-1.4em}
    \begin{columns}[T,totalwidth=\textwidth]
        \tcbset{equal height group=sp}%
        \begin{column}{0.50\textwidth}
            \begin{tceqbox}[1]{}
             $|\phi_k^\lambda\rangle = \sum_{p=\pm} |s_p\rangle |\phi_{k,p}^\lambda\rangle \in \mathbb{C}^2 \otimes \mathcal{H}_\lambda$.
            \end{tceqbox}
        \end{column}
        \hfill
        \begin{column}{0.48\textwidth}
            \begin{tceqbox}[1]{}
                \vspace{-1.2em}
                \begin{align*}
                    \ket{s_\pm} \xrightarrow{\sigma} \pm \ket{s_\pm} \quad \text{for } \sigma \in S_n \setminus A_n
                \end{align*}
            \end{tceqbox}
        \end{column}
    \end{columns}
    \begin{tceqbox}[1]{}
        \vspace{-0.5em}
        \begin{equation*}
            \text{Tr}\left(\rho_0^{(\text{ii})}\rho_1^{(\text{ii})}\right) = \sum_{\lambda \in R_{\text{ii}}} \sum_{k,l=1}^{d_\lambda} \frac{\left| \braket{\phi_{k,+}^\lambda|\phi_{l,+}^\lambda} - \braket{\phi_{k,-}^\lambda|\phi_{l,-}^\lambda} \right|^2}{d_\lambda}
        \end{equation*}
    \end{tceqbox}
    \begin{tceqbox}[1]{}
                $\implies$ Supports are orthogonal if $\braket{\phi_{k,+}^\lambda|\phi_{l,+}^\lambda} = \braket{\phi_{k,-}^\lambda|\phi_{l,-}^\lambda}$ for all $\lambda, k, l$.
    \end{tceqbox}
\end{frame}
\begin{frame}{Proof (II): Action of $A_n$ and Block Structure}
    \begin{oceqbox}[2]{\smartball[WLTeal]{3} } Restriction has no effect ($\mathcal{K}_\lambda \downarrow_{A_n} = \mathcal{K}_\lambda$).
    \begin{equation*}
        \rho_0^{(\text{iii})} = \sum_{\lambda \in R_{\text{iii}}} \frac{\mathds{1}_\lambda}{d_\lambda} \otimes \Phi^\lambda
    \end{equation*}
    \end{oceqbox}
    \begin{outlinebox}[WLBlue]
        \begin{center}
        $\rho_0^{(\text{iii})}$ is invariant under all of $S_n$, so $\rho_0^{(\text{iii})} = \rho_1^{(\text{iii})}$.
        \end{center}
    \end{outlinebox}
\end{frame}

% ------------------------------------------------------------------------
% Slide 3: Conditions for Distinguishability & Conclusion
% ------------------------------------------------------------------------
\begin{frame}{Proof (III): Distinguishability Conditions}
    \begin{stylebox}[1]{$d \ge d_{\min} := \lceil \sqrt{n} \rceil$}
            \smartball{1} and/or \smartball{2} hold, enabling perfect distinguishability.
    \end{stylebox}
    \begin{stylebox}[4]{$d \le d_{\min} := \lceil \sqrt{n} \rceil$}
            \smartball{1} and \smartball{2} fail,only \smartball{3} remains
            \begin{center}
            $\rho_0 = \rho_1$ and $P_s = 1/2$.
            \end{center}
    \end{stylebox}
    \begin{columns}[T]
        \begin{column}{0.50\textwidth}
            \begin{tceqbox}[1]{}
                This geometrically corresponds to the impossibility of fitting certain Young Diagrams within a $d \times d$ box when $d$ is too small.
            \end{tceqbox}
        \end{column}
        \begin{column}{0.40\textwidth}
            \vspace{-1em}
            \begin{center}
                \includegraphics[width=0.8\textwidth]{pic/constraint_box.pdf}
            \end{center}
        \end{column}
    \end{columns}
\end{frame}
% ------------------------------------------------------------------------
% Slide: Entanglement Requirement
% ------------------------------------------------------------------------
\begin{frame}{Entanglement Requirement for Parity Identification}
    \begin{stylebox}[1]{Question}
                How entangled must a parity-detecting state be?
            \end{stylebox}
    \vspace{-1.5em}
    \begin{columns}[T,totalwidth=\textwidth]
        \begin{column}{0.54\textwidth}
            \begin{tceqbox}[1]{Metric:geometric measure of entanglement(GME)}
                \vspace{-1em}
                \begin{equation*}
                    E(\ket{\psi}) = 1 - \max_{\ket{\phi}\in \text{SEP}} |\braket{\phi|\psi}|^2
                \end{equation*}
            \end{tceqbox}
            \begin{tceqbox}[1]{Minimum Required Entanglement:}
                \vspace{-1em}
                    \begin{equation*}
                        E_{\lambda a} = 1 - \max_{\ket{\phi} \in \text{SEP}} \bra{\phi}\mathcal{P}_{\lambda a}\ket{\phi}
                    \end{equation*}
                    This lower bounds the entanglement of any parity-detecting state supported on this subspace.
            \end{tceqbox}
        \end{column}
        \begin{column}{0.45\textwidth}

            \begin{table}
                \centering
                \small % 稍微缩小表格字体以适应页面
                \caption{Minimum required ($E_{\lambda a}$) vs. known maximal ($E_{\max}$) entanglement.}
                \begin{tabular}{ccc}
                    \hline \hline
                    $n$ & $E_{\lambda a}$ & $E_{\max}$ \\
                    \hline
                    3 & $5/9$ & $5/9^\dagger$ \\
                    4 & $7/9$ & $7/9^\dagger$ \\
                    5 & $17/20$ & $\approx 0.96$ \\
                    \hline \hline
                \end{tabular}
                \begin{footnotesize}
                \newline $\dagger$ indicates proven maxima.
                \end{footnotesize}

            \end{table}
            \begin{tceqbox}[1]{Observation:}
                maximal for $n=3, 4$, and close to the maximum for $n=5$.
            \end{tceqbox}
        \end{column}

    \end{columns}

\end{frame}
\begin{frame}{Summary and Outlook}
    \begin{columns}[T,totalwidth=\textwidth]
        \tcbset{equal height group=s1}%
        \begin{column}{0.49\textwidth}
            \begin{tceqbox}{A Sharp Quantum Advantage}
                Achievable with $d = \lceil \sqrt{n} \rceil$.
            \end{tceqbox}
        \end{column}
        \hfill
        \begin{column}{0.49\textwidth}
            \begin{tceqbox}{}
                    Relies purely on permutation symmetry without invoking specific dynamics.
            \end{tceqbox}
        \end{column}
    \end{columns}
    \begin{columns}[T,totalwidth=\textwidth]
        \tcbset{equal height group=s2}%
        \begin{column}{0.49\textwidth}
            \begin{tceqbox}{}
                Requires \textbf{no ancillas} (increasing multiplicities does not help).
            \end{tceqbox}
        \end{column}
        \hfill
        \begin{column}{0.49\textwidth}
            \begin{tceqbox}{}
                    Demands entanglement close to the theoretical maximum.
            \end{tceqbox}
        \end{column}
    \end{columns}
    \begin{columns}[T,totalwidth=\textwidth]
        % \tcbset{equal height group=s3}%
        \begin{column}{0.49\textwidth}
            \begin{tceqbox}[5]{Practical Implementations:}
                \begin{itemize}
                    \item State preparation via experimentally accessible Hamiltonians.
                    \item Designing specific quantum circuits and measurements.
                \end{itemize}
            \end{tceqbox}
        \end{column}
        \hfill
        \begin{column}{0.49\textwidth}
            \begin{tceqbox}[5]{Theoretical Extensions:}
                \begin{itemize}
                    \item Assessing robustness of the $\sqrt{n}$ advantage under noise.
                    \item Exploring other symmetry groups.
                \end{itemize}
            \end{tceqbox}
        \end{column}
    \end{columns}
\end{frame}


% --- Slide 12: End ---
% \ThankYouPage

\end{document}